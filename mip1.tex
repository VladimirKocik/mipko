\documentclass[10pt,twoside,slovak,a4paper]{article}
\usepackage[slovak]{babel}
%\usepackage[T1]{fontenc}
\usepackage[IL2]{fontenc}
\usepackage[utf8]{inputenc}
\usepackage{graphicx}
\usepackage{url} % príkaz \url na formátovanie URL
\usepackage{hyperref}

\usepackage{cite}

\pagestyle{headings}


\title{Rozpoznávanie objektov v reálnom čase pre autonómne vozidlá\thanks{Semestrálny projekt v predmete Metódy inžinierskej práce, ak. rok 2021/22}}
\author{Vladimír Kočík\\[2pt]
	{\small Slovenská technická univerzita v Bratislave}\\
	{\small Fakulta informatiky a informačných technológií}\\
	{\small \texttt{xkocik@stuba.sk}}
	}

\date{November 2021}



\begin{document}
\maketitle
\begin{abstract}

Vývoj autonómnych vozidiel neustále pokračuje, a každým dňom sa stávajú metódy využívané samoriadiacimi autami rýchlejšie a zároveň bezpečnejšie. Aj keď dnes je o komerčnom využití týchto vozidiel ešte priskoro uvažovať, myslím že o niekoľko rokov sa to zmení. Jedna z motivácií je znížiť počet dopravných nehôd, keďže dnes je väčšina z nich spôsobená práve chybou alebo nepozornosťou vodiča. V mojom článku som sa chcel zamerať na metódy využívané autonómnymi vozidlami na rozpoznávanie hranice cesty a objektov na ceste a ako môže vyzerať plánovanie bezpečného zaraďovania auta vo dvojprúdových a trojprúdových cestách.

%Keďže najnovší výskum nie je pravidelne publikovaný a sprístupnený verejnosti, je mnou opísané metódy sú dnes nahradené lepšími a efektivnejšími, no stále majú hodnotu pri študovaní vývoja týchto technológií. Samozrejme to nie je niečo, z čoho by sa každý dokázal odborník po prečítaní pár článkov, aj moje porozumenie témam je limitované, no mojou snahou bude čo najbližšie vysvetliť danú tému.

\end{abstract}

\section{Úvod} 
Pri autonómnych vozidlách je kľúčové presné zaznamenanie prostredia, lokalizácia, mapovanie a iné technológie ktoré čerpajú dáta z kamery, taktiež GPS, IMU(Inertial Measurement Unit), odometria kolies, senzory LIDAR(Light Detection and Ranging) 


\cite{6629552}

\section{Rozpoznávanie čiar na ceste}
\cite {KSII}
\cite {9179748}
\cite{tiis:22283}
\cite{tiis:23914}


\section{Rozpoznávanie objektov}


\cite {9337402}
\cite {9253253}

\section{Plánovanie - bezpečné zaraďovanie}


\cite {9034121}

\section{Záver}

\bibliography{zdroje}
\bibliographystyle{plain}

\end{document}